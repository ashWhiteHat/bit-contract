\section{Appendix}

\subsection{A1}
本稿の第3章では、identity-opaque marketplace with custodial escrow(IOM-CE)という抽象モデルを用い、資金の一時保有(custody)に起因する custodial misappropriation / marketplace disappearance risk を問題化した。IOM-CEは特定領域の運用手順を記述するためではなく、身元が確証しにくい環境で市場取引を反復可能にする制度要素を、先行研究の反復観察に基づき最小共通集合として抽出するための分析単位である。

本付録は、主要先行研究において、IOM-CEの構成要素がどの程度明示されているかを整理し、本文のモデル化が恣意的でないことを示す。なお、対象文献には高リスク環境に関する計測研究が含まれるが、本稿はそれらを設計対象として推奨・再現する意図を持たず、抽象モデルの経験的根拠としてのみ参照する。

\subsection{A2}
本節では、IOM-CE構成要素の出現状況を先行研究ごとに整理する。

\begin{table}[htbp]
	\centering
	\footnotesize
	\begin{tabular}{lcccc}
		\toprule
		要素(IOM-CE)                                & Christin (2013) & Soska \& Christin (2015) & Tzanetakis et al.\ (2016) & Spagnoletti et al.\ (2022) \\
		\midrule
		Buyer/Seller(需要・供給ロール)                    & \checkmark      & \checkmark               & \checkmark                & \checkmark                 \\
		Operator/Custodian(運営/資金保有主体)             & \checkmark      & \checkmark               & \checkmark                & \checkmark                 \\
		Escrow / Payment facilitation(エスクロー等)     & \checkmark      & \checkmark               & \checkmark                & \checkmark                 \\
		Reputation / Feedback(評判・レビュー)            & \checkmark      & \checkmark               & \checkmark                & \checkmark                 \\
		Disappearance / Scams / Breakdown(消失・詐欺等) & \checkmark      & \checkmark               & \checkmark                & \checkmark                 \\
		Resilience / Migration(移行・存続形態)           & \checkmark      & \checkmark               & 必要に応じて                    & \checkmark                 \\
		\bottomrule
	\end{tabular}
\end{table}
