\section{Formalization}
本章では、UTXO集合を状態としトランザクションを遷移とするステートマシンとしてBitcoinを再定式化する。
その上で、スマートコントラクトを「資産の拘束($\mathrm{lock}$)」「解放条件($\text{unlock predicate}$)」「条件成立時に許される出力($\text{prescribed transition/template}$)」として抽象化し、UTXOに新たな状態型としてLocked UTXOを導入する。
Locked UTXOは、単なるスクリプト断片ではなく、テンプレート参照と決定的な検証規則を伴う型付き状態として扱われ、これにより監査可能性が上限付きの契約表現を可能にする。
次章では、この形式化を具体的なテンプレート($\mathrm{deposit}/\mathrm{release}/\mathrm{refund}$)へ具体化し、必要な変数・パラメータ・検証規則を与える。

\subsection{目的とスコープ}

本章の目的は、Bitcoin を UTXO 集合を状態、トランザクションを遷移とするステートマシンとして再定式化し、その上でスマートコントラクトを (1) 資産の拘束(lock)(2) 解放条件(unlock predicate)(3) 条件成立時に許される出力(prescribed transition / template)として形式的に定義することである。

本稿は 汎用 VM による任意計算と mutable global state(共有可変ストレージ)を採用しない。
代わりに、コントラクトは テンプレート(template)として事前に制約され、インスタンス化はテンプレート参照と引数によって与えられる。
共有状態は最小化され、必要であればテンプレート集合や参照情報は immutable global registry(不変レジストリ)として保持し得る(R4 に整合)。

本章は「一般形式(general form)」までを与え、ユースケース適用は次章でテンプレート群(deposit/release/refund)として具体化する。

\subsection{UTXO ステートマシン}
\paragraph{定義 1(UTXO と Outpoint)}
Outpoint を $\mathsf{op}=(\mathsf{txid}, i)$ とし、UTXO を次で表す:
\[
	u=(\mathsf{op}, v, \mathsf{lockScript}, \mathsf{data})
\]
ここで $v$ は額面、$\mathsf{lockScript}$ は消費条件、$\mathsf{data}$ は(本稿で導入する)テンプレート参照や状態タグ等の付随データである。
(Bitcoin Script 相当では $\mathsf{data}$ を script/commitment/metadata として埋め込む。eUTXO では datum として扱える。)

\paragraph{定義 2(台帳状態)}
台帳状態を
\[
	\Sigma := (U, \Gamma)
\]
とする。$U$ は未消費 UTXO の集合、$\Gamma$ はテンプレート集合(および必要な参照情報)から成る不変レジストリである(R4)。

\paragraph{定義 3(トランザクション)}
トランザクション $tx$ を
\[
	tx := (\mathsf{in}, \mathsf{out}, \mathsf{wit})
\]
とする。$\mathsf{in}$ は入力 outpoint 列、$\mathsf{out}$ は出力 UTXO 列、$\mathsf{wit}$ は署名等の witness である。

\paragraph{定義 4(基礎検証:UTXO 妥当性)}
$tx$ が $\Sigma$ 上で基礎的に妥当であるとは、次を満たすことをいう。
\begin{enumerate}
	\item[(i)] 入力が $U$ の要素を参照し、二重消費がない。
	\item[(ii)] 各入力の $\mathsf{lockScript}$ が $\mathsf{wit}$ により満たされる。
	\item[(iii)] 価値保存($\sum v_{\mathrm{in}}=\sum v_{\mathrm{out}}+\mathsf{fee}$)等のプロトコル規則を満たす。
\end{enumerate}

\subsection{スマートコントラクトの抽象定義:Lock / Predicate / Prescribed Transition}
本稿は、スマートコントラクトを「汎用プログラム」ではなく、条件付き資産移転として抽象化する。

\paragraph{定義 5(テンプレート型コントラクト)}
テンプレート型スマートコントラクトを次の 3 つ組として定義する:
\[
	C := (\mathsf{Lock}, \mathsf{Pred}, \mathsf{Trans})
\]
\begin{itemize}
	\item $\mathsf{Lock}$:資産を拘束する操作(どの形の UTXO を Locked 状態にするか)
	\item $\mathsf{Pred}$:解放条件(unlock predicate)。外部裁量ではなく検証可能な述語である(R2)。
	\item $\mathsf{Trans}$:条件成立時に許される出力形(prescribed transition / template)。出力が任意ではなく、テンプレートで上限付きに規定される(R3)。
\end{itemize}

\subsection{オンチェーン対象:Template / Contract Instance / Locked UTXO}
本稿の対象を、“状態として何が増えるか”が明確になるように、次の 3 層に整理する。

\paragraph{定義 6(Template:普遍)}
テンプレート $\tau \in \Gamma$ は次の構造を持つ:
\[
	\tau := (\mathsf{tid}, \mathsf{ArgSchema}, \mathsf{Logic}, \mathsf{TransSpec}, \mathsf{Bound})
\]
\begin{itemize}
	\item $\mathsf{tid}$:テンプレート ID(固定)
	\item $\mathsf{ArgSchema}$:引数の型・制約
	\item $\mathsf{Logic}$:Computational Law(例:Stipula)等で記述された検証ロジック
	\item $\mathsf{TransSpec}$:許される状態遷移ラベルと、それぞれの出力仕様
	\item $\mathsf{Bound}$:監査上限を規定する境界(コードサイズ、分岐数、参照数など)
\end{itemize}
テンプレートは \textbf{普遍(protocol-level constant)} であり、台帳状態の更新で変化しない(R3, R4)。

\paragraph{定義 7(Contract Instance:インスタンス記述子)}
コントラクト・インスタンス $I$ は
\[
	I := (\mathsf{cid}, \mathsf{tid}, \vec{a}, \mathsf{participants})
\]
で表す。$\vec{a}$ は初期化引数、$\mathsf{participants}$ は(必要なら)当事者集合(例:Buyer/Seller/Operator)である。
$\mathsf{cid}$ は決定的に導出される識別子であり、例えば
\[
	\mathsf{cid} := H(\mathsf{createTxid} \parallel i \parallel \mathsf{tid} \parallel \vec{a})
\]
のように作れる(重要なのは同一入力から一意であること)。これにより「参照先インスタンスが曖昧」「後から差し替え」等の裁量余地を縮退する(R2, R3)。

\paragraph{定義 8(Locked UTXO:型付き状態)}
Locked UTXO は、単なる script 断片ではなくテンプレート参照と状態タグを伴う型付き状態である、という設計選好を仕様として固定する。
Locked UTXO を
\[
	u^\star := (\mathsf{op}, v, \mathsf{lockScript}, \mathsf{cid}, \mathsf{stateTag})
\]
と表す。$\mathsf{stateTag}$ はテンプレートが定める有限個の状態ラベル(例:$S0, S0D, S0X$)である。
これにより「Script の中に条件を書く」から、「状態(タグ/メタ)に条件を載せ、テンプレート参照で検証する」へと、監査可能性が上限付きの表現に移る。

\subsection{遷移規則:Create / Lock / Release / Finalize と Admissibility}
本節では、Locked UTXO を生成・消費する遷移の妥当性を admissibility(許容条件)として定義する。要件は「外部裁量ではなく検証可能」=決定的 admissibility である(R2)。

\paragraph{定義 9(Contract Create / Lock)}
\begin{itemize}
	\item CreateSC:$\Gamma$ 内のテンプレート $\mathsf{tid}$ と引数 $\vec{a}$ を参照し、$\mathsf{cid}$ を生成するトランザクション。
	\item Lock:通常 UTXO $u$ を消費し、Locked UTXO $u^\star$ を出力する。出力には $(\mathsf{cid}, \mathsf{stateTag}=s_{\mathrm{init}})$ が付与される。
\end{itemize}

\paragraph{定義 10(Release / Transition)}
Locked UTXO $u^\star$ を入力として消費するトランザクションを contract transition tx と呼ぶ。
このとき、テンプレート $\tau$ が定める遷移ラベル $\ell$ に対し、次を満たすときに限り $tx$ は admissible とする。
\begin{enumerate}
	\item[(i)] 述語の成立:$\mathsf{Pred}_{\tau,\ell}(\Sigma, tx, \mathsf{cid}) = \mathsf{true}$
	\item[(ii)] 出力の一致:$\mathsf{out}$ が $\mathsf{TransSpec}_{\tau,\ell}$ に一致(受益者・配分・ロック継続・次状態タグ等)
\end{enumerate}

\paragraph{定義 11(Finalize / Timeout)}
テンプレートは「誰でも起動可能な確定遷移」を持ち得る。たとえば、期限経過により自動的に決着する遷移($\operatorname{after}(d), \operatorname{after}(t)$)を $\ell_{\mathrm{final}}$ として与え、上の admissibility で検証可能にする。これにより資金が永久拘束されないこと(liveness)を仕様として扱える(RQ3)。

\subsection{不変条件(Invariants)と要件 R1--R4 への対応}
本稿は、資金拘束・解放を「人間の裁量」ではなく制度(検証規則)として固定し、失敗遷移(misappropriation)を排除/拘束することを狙う。
以下、テンプレート型スマートコントラクトが満たすべき不変条件を定義し、R1--R4 と対応付ける(RQ3)。

\paragraph{Invariant S(Safety:価値保全)}
任意の admissible $tx$ に対し、価値保存と受益者制約が破れない。
\begin{itemize}
	\item 価値が任意に増えない(base rule)
	\item テンプレートが許さない受益者への移転が生じない($\mathsf{TransSpec}$ により拘束)
\end{itemize}

\paragraph{Invariant A(Authority Safety:単独カストディ排除)}
Operator が単独で資金を不正流用できる遷移 $F$ が admissible にならない(R1)。
典型には、署名条件・共同署名・期限付き自動遷移などで「単独での解放」を制度的に排除する。

\paragraph{Invariant L(Liveness:決着性)}
テンプレートが定める前提(期限・当事者署名など)が満たされれば、Locked 状態が永久に残留しない(デッドロックしない)。これは「Finalize/Timeout が誰でも起動可能」等で与えられる(RQ3)。

\paragraph{Invariant B(Bounded Auditability:監査上限)}
各テンプレート $\tau$ について、検証者が点検すべきロジックは $\mathsf{Bound}$ により上限付けられ、インスタンス化により複雑性が無制限に増えない(R3)。

\paragraph{要件対応(まとめ)}
\begin{itemize}
	\item R1:$F$ を admissibility から排除(Authority Safety)
	\item R2:$\mathsf{Pred}$ と $\mathsf{TransSpec}$ による決定的 admissibility
	\item R3:Template + $\mathsf{Bound}$ による監査上限
	\item R4:状態は基本的に UTXO(局所状態)。共有は $\Gamma$(不変)に限定
\end{itemize}

\subsection{次章への接続(テンプレート具体化)}
次章では、本章の一般形式を、焦点ケース(IOM-CE)に対応する具体テンプレート($\mathrm{deposit}/\mathrm{release}/\mathrm{refund}$)へ具体化し、状態ラベル、引数、署名条件、期限条件、出力仕様を $\mathsf{TransSpec}$ として与える。
