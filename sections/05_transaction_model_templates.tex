\section{Transaction Model}
本章では、4章の形式($\mathsf{Pred}$ / $\mathsf{TransSpec}$)に合わせて、IOM-CEテンプレートの遷移仕様を正規化して記述する。

\subsection{Roles and Variables}
本章で用いる役割と変数を先に固定する。
\begin{itemize}
	\item Buyerアドレス: $A$
	\item Sellerアドレス: $B$
	\item Operatorアドレス: $C$
	\item 商品価格: $p$
	\item マージン率: $m$
	\item 検収猶予(配送後の待機時間): $d$
	\item 最終タイムアウト: $t$($t \ge d$)
\end{itemize}

\subsection{State Model and Transitions}
本テンプレートの状態ラベルは $S0, S0D, S0X, S1, S2$ とし、意味は次である。
\begin{itemize}
	\item $S0$: Funded escrow(資金拘束)
	\item $S0D$: Delivered pending(配送通知済み・検収猶予中)
	\item $S0X$: Disputed(紛争中)
	\item $S1$: Release to Seller(支払)
	\item $S2$: Refund to Buyer(返金)
\end{itemize}

遷移の全体像は次のように一枚で要約できる。
\begin{itemize}
	\item $\mathrm{BuyerLock}$ で資金を拘束し $S0$ を開始。
	\item $S0 \to S0D$($\mathrm{SellerDeliver}$)、$S0D \to S1$($\mathrm{BuyerConfirm}$ または $\operatorname{after}(d)$)。
	\item $S0D \to S0X$($\mathrm{BuyerDispute}$)、$S0X \to S1/S2$(仲裁)。
	\item 取消およびタイムアウトにより $S0/S0D/S0X \to S2$ が可能。
\end{itemize}

\subsection{Transition Catalog}
各遷移ラベル $\ell$ に対する $\mathsf{Pred}_{\tau,\ell}$ と $\mathsf{TransSpec}_{\tau,\ell}$ を、4章の形式に沿って列挙する。表中の不変条件は S=Safety, A=Authority Safety, L=Liveness を表す。取消フラグは Locked UTXO の $\mathsf{data}$ に記録される補助情報である。
\begin{table}[htbp]
	\centering
	\footnotesize
	\setlength{\tabcolsep}{3pt}
	\renewcommand{\arraystretch}{1.15}
	\begin{tabular}{p{0.14\textwidth}p{0.08\textwidth}p{0.19\textwidth}p{0.23\textwidth}p{0.12\textwidth}p{0.12\textwidth}}
		\toprule
		遷移ラベル $\ell$ & 前提状態 & $\mathsf{Pred}_{\tau,\ell}$ が見る要素 & $\mathsf{TransSpec}_{\tau,\ell}$(出力仕様) & ブロードキャスト & 不変条件 \\
		\midrule
		BuyerLock & -- & $\mathrm{sig}(A)$, $p$ & $p$ を拘束し、$\mathrm{state}=S0$ の Locked UTXO を生成 & Buyer & S \\
		SellerDeliver & $S0$ & $\mathrm{sig}(B)$ & $\mathrm{state}=S0D$ に更新(ロック継続) & Seller & S, L \\
		BuyerConfirm & $S0D$ & $\mathrm{sig}(A)$ & $p(1-m)$ を $B$、$p \cdot m$ を $C$ へ & Buyer & S, A, L \\
		BuyerDispute & $S0D$ & $\mathrm{sig}(A)$, $\operatorname{before}(d)$ & $\mathrm{state}=S0X$ に更新(ロック継続) & Buyer & A, L \\
		BuyerCancel & $S0/S0D$ & $\mathrm{sig}(A)$ & 取消フラグを付与(ロック継続) & Buyer & A \\
		SellerAcceptCancel & $S0/S0D$ & $\mathrm{sig}(B)$, 取消フラグ & $p$ を $A$ に返金($S2$) & Seller & S, A, L \\
		OperatorMediateRelease & $S0X$ & $\mathrm{sig}(C)$ と $(\mathrm{sig}(A)$ or $\mathrm{sig}(B))$ & $p(1-m)$ を $B$、$p \cdot m$ を $C$ へ & Operator + A/B & S, A, L \\
		OperatorMediateRefund & $S0X$ & $\mathrm{sig}(C)$ と $(\mathrm{sig}(A)$ or $\mathrm{sig}(B))$ & $p$ を $A$ に返金($S2$) & Operator + A/B & S, A, L \\
		ContractFinalize & $S0D$ & $\operatorname{after}(d)$ & $p(1-m)$ を $B$、$p \cdot m$ を $C$ へ & Anyone & S, L \\
		ContractTimeout & $S0/S0D/S0X$ & $\operatorname{after}(t)$ & $p$ を $A$ に返金($S2$) & Anyone & S, L \\
		\bottomrule
	\end{tabular}
\end{table}
